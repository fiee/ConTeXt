%D \module
%D    [          file=t-lilypond,
%D            version=2010.03.11,
%D              title=\CONTEXT\ User Modules,
%D           subtitle=LilyPond Bridge (Music Typesetting),
%D             author={Christopher Creutzig, H. Ramm et al.}, % with contributions by Mojca Miklavec and Wolfgang Schuster
%D		   maintainer={Henning Hraban Ramm, hraban@fiee.net},
%D               date=\currentdate,
%D          copyright=the authors]
%C
%C This module is licensed under the GNU GPL V2.

%D This module serves to include LilyPond music directly in the
%D \CONTEXT\ source, just as \METAPOST\ code can be.
%D
%D It is supposed to work only with LilyPond's stable version 2.12 and \CONTEXT\ MkIV i.\,e. \LUATEX.
%D
%D {\bfa Complete list of options:}
%D
%D \starttabulate[|l|l|l|]
%D    \NC option\NC default\NC comment\NC\NR
%D    \HL
%D    \NC staffsize\NC 14\NC in pt    \NC\NR
%D    \NC indent\NC 0pt\NC first line indent    \NC\NR
%D    \NC time\NC yes\NC count time? (i.e. show time signature, set bars, split line)    \NC\NR
%D    \NC clef\NC yes\NC show clef?    \NC\NR
%D    \NC linewidth\NC \type{\localhsize}\NC    \NC\NR
%D    \NC betweensystemspace\NC 54pt\NC space between systems    \NC\NR
%D    \NC height\NC \type{\textheight}\NC    \NR
%D    \NC align\NC ?\NC default depends on fragment=...    \NC\NR
%D    \NC fragment\NC no\NC whole staff line or only snippet?   \NC\NR
%D    \NC barnumbers\NC no\NC show numbers of measures?    \NC\NR
%D    \NC showempty\NC no\NC drop empty staves?    \NC\NR
%D    \NC seriffont\NC "TeX Gyre Schola"\NC font for e.g. lyrics    \NC\NR
%D    \NC sansfont\NC "LMSans10"\NC font for e.g. chords    \NC\NR
%D    \NC monofont\NC "LMTypewriter10" \NC normally not used    \NC\NR
%D    \NC tmpdir\NC ./lilytemp \NC directory for temporary files \NC\NR
%D    \NC imgdir\NC ./lilypdfs \NC directory for image files (note PDFs) \NC\NR
%D \stoptabulate
%D
%D {\bfa Todo list}
%D
%D \startitemize[packed,joinedup]
%D \item Proper support for multipage results.
%D \item Set LilyPond page size to remaining page space.
%D \item Place instrument names in margin.
%D \item Work with existing PDFs on systems where LilyPond isn't installed.
%D \stopitemize
%D
%D {\bfa Commented Source}
%D
%D As usual, we use a prefix for buffers, \type{\getparameters} etc.:

\writestatus{loading}{LilyPond module}

\startmodule[lilypond]

\unprotect

\definesystemvariable{lily}

%D Define the text snippets to be placed around fragments.
%D Since LilyPond uses similar syntax to TeX, we must do some catcode fiddling.

\bgroup

\catcode`\/=\@@escape
/catcode`/\=/@@other
/catcode`/#=/@@other
/catcode`/<=/@@begingroup
/catcode`/>=/@@endgroup
/catcode`/{=/@@other
/catcode`/}=/@@other
/catcode`/%=/@@other
/catcode`/-=/@@letter

/gdef/lily!fragmentprefix
	</string^^J
	 /string^^J
	 {/string^^J
	 % ly snippet contents follows:/string^^J
	>

/gdef/lily!fragmentsuffix
	</string^^J
	 % end ly snippet/string^^J
	 }/string^^J
	>

/gdef/lily!prefix
	</string^^J
	 /string^^J
	 #(ly:set-option (quote no-point-and-click))/string^^J
	 #(ly:set-option (quote delete-intermediate-files))/string^^J
	 /string^^J
	 \version "2.12.0"/string^^J
	 #(define version-seen? #t)/string^^J
	 \layout {/string^^J
	 /iflily!showempty/else
	   % drop empty staves/string^^J
	   \context { \RemoveEmptyStaffContext } % MUST come first/string^^J
	   \context { \Score \override VerticalAxisGroup #'remove-first = ##t }/string^^J
	 /fi
	 /string^^J
	 /unless/iflily!time
	   % switch off time signature (i.e. only one line, no bars)/string^^J
	   ragged-last = ##t/string^^J
	   \context { \Score timing = ##f }/string^^J
	   \context { \Staff \remove "Time_signature_engraver" }/string^^J
	   /string^^J
	 /fi
	 /unless/iflily!barnumbers
	   % switch off bar numbers/string^^J
	   \context { \Score \remove "Bar_number_engraver" }/string^^J
	 /fi
	 /string^^J
	 /unless/iflily!clef
	   % switch off clef sign/string^^J
	   \context { \Staff \remove "Clef_engraver" }/string^^J
	 /fi
	 /string^^J
	 #(define fonts
	   (make-pango-font-tree
	   /@@lilyseriffont/space
	   /@@lilysansfont/space
	   /@@lilymonofont/space
	 (/letterslash/space/@@lilystaffsize/space 20)))/string^^J
	 /string^^J
	 }/string^^J
	 \paper {/string^^J
	   #(define dump-extents #t)/string^^J
	   ragged-right = /iflily!align ##f/else ##t/fi/string^^J
	   indent = /withoutpt/the/lily!indent\pt/string^^J
	   line-width = /withoutpt/the/lily!linewidth\pt/string^^J
	   vsize = /withoutpt/the/lily!vsize\pt/string^^J
	   printpagenumber = ##f/string^^J
	   oddFooterMarkup=##f/string^^J
	   oddHeaderMarkup=##f/string^^J
	   bookTitleMarkup = ##f/string^^J
	   scoreTitleMarkup = ##f/string^^J
	   ragged-bottom=##t/string^^J
	   ragged-last-bottom=##t/string^^J
	   between-system-padding = #0/string^^J
	   between-system-space = /withoutpt/the/lily!betweensystemspace\pt/string^^J
	 }/string^^J
	>

/egroup

%D Here, we define our parameters:

\newdimen\lily!linewidth
\newdimen\lily!vsize
\newdimen\lily!indent
\newdimen\lily!betweensystemspace
\newif\iflily!showempty
\newif\iflily!align
\newif\iflily!clef
\newif\iflily!time
\newif\iflily!barnumbers

\newcounter\lily!figures

%D Again, as usual, there is a \type{\setuplilypond} command
%D that accepts the same parameters as \type{\lilypond} and
%D \type{\startlilypond} do in their optional argument.

\def\setuplilypond
	{\bgroup
	 \catcode`\#=\@@other
	 \dosingleempty\dosetuplilypond}

\def\dosetuplilypond[#1]%
	{\egroup
	 \getparameters[\??lily][#1]}

%D We set the following defaults:

\setuplilypond
	[staffsize=20,
	 indent=0pt,
	 betweensystemspace=54pt,
	 time=\v!yes,
	 clef=\v!yes,
	 align=?,	% default depends on fragment=...
	 fragment=\v!no,
	 barnumbers=\v!no,
	 showempty=\v!no,
	 seriffont="TeX Gyre Schola", % LilyPond default font
	 sansfont="LMSans10",
	 monofont="LMTypewriter10",
	 tmpdir={./lilytemp},
	 imgdir={./lilypdfs}]
%D
%D If you want to know which fonts of your system LilyPond can see
%D with which names, try \type{lilypond -dshow-available-fonts any}.
%D (The 3rd parameter is necessary, but can be anything.)\par
%D There are a lot of restrictions, e.g. no fonts from the \TeX\ tree (don't know why),
%D only one face per font file etc.

%D
%D Check if temporary and image directories exist and create them if not:
%D

\def\tmpdir{\@@lilytmpdir} % for intermediate files
\def\imgdir{\@@lilyimgdir} % for music PDFs

\startluacode	
	if not lfs.isdir("\tmpdir") then
		logs.report("LilyPond Warn ", "Temporary directory not found, trying to make '\tmpdir'")
		lfs.mkdir("\tmpdir")
	end
	if not lfs.isdir("\imgdir") then
		logs.report("LilyPond Warn ", "Image directory not found, trying to make '\imgdir'")
		lfs.mkdir("\imgdir")
	end
\stopluacode

%D
%D \type{\startlilypond} is a multistage implementation, because
%D end-of-line characters must be treated specially in the
%D \type{\startlilypond} \dots\ \type{\stoplilypond} range.
%D

\def\startlilypond
	{\bgroup
	 \catcode`\#=\@@other
	 \dosingleempty\dostartlilypond}

\def\dostartlilypond[#1]%
	{\obeylines
	 \dodostartlilypond[#1]}

\long\def\dodostartlilypond[#1]#2\stoplilypond
	{\egroup
	 \bgroup
%D
%D The default of the \type{linewidth} parameter is the local \type{\hsize}.
%D
	 \setlocalhsize
	 \getparameters[\??lily][linewidth=\the\localhsize,height=\the\textheight,#1]%
	 \lily!linewidth\@@lilylinewidth
	 \lily!vsize\@@lilyheight
	 \lily!indent\@@lilyindent
	 \lily!betweensystemspace\@@lilybetweensystemspace
%D
%D The default of \type{align} depends on whether we typeset a fragment:
%D
	 \doifelse\@@lilyalign\v!yes\lily!aligntrue\lily!alignfalse
	 \doif\@@lilyfragment\v!no{\doif\@@lilyalign{?}\lily!aligntrue}%
	 \doifelse\@@lilytime\v!yes\lily!timetrue\lily!timefalse
	 \doifelse\@@lilybarnumbers\v!yes\lily!barnumberstrue\lily!barnumbersfalse
	 \doifelse\@@lilyshowempty\v!yes\lily!showemptytrue\lily!showemptyfalse
	 \doifelse\@@lilyclef\v!yes\lily!cleftrue\lily!cleffalse
%D
%D We are using a counter to keep the different LilyPond pieces
%D separate. This allows to typeset them only once, during the
%D first run.
%D
%D TODO: This won't work any longer once we pass the remaining
%D vertical space to LilyPond.
%D
	 \doglobal\increment\lily!figures
	 \edef\lily!filename{\bufferprefix lilypond-\lily!figures}%
	 \edef\lily!tmpfilename{\tmpdir/\lily!filename.tmp}%
	 \edef\lily!img{\imgdir/\lily!filename.pdf}
	 \def\obeyedlines{\string^^J}%
	 \convertargument#2\to\ascii
	 \startexpanded
	   \noexpand\setbuffer[\lily!tmpfilename]%
	   \lily!prefix
	   \letterhash(set-global-staff-size \getvalue{\??lily staffsize})\string^^J%
	   \ifdefined\@@lilyfragment\lily!fragmentprefix\fi
%D TODO: Option \quote{packed}
	   \ascii
	   \ifdefined\@@lilyfragment\lily!fragmentsuffix\fi
	   \noexpand\endbuffer
	 \stopexpanded
%D
%D In \LUATEX\ buffers are kept in memory
%D in contrast to older flavours of \TeX\ where they're written to a
%D file automatically. We have to require writing to a file explicitely
%D with \type{\savebuffer}.
%D
	\writestatus{LilyPond}{\lily!filename}
	\startluacode
		buffers.save('\lily!tmpfilename')
		if not lfs.isfile('\lily!tmpfilename') then
			logs.report("LilyPond ERROR", "File '\lily!tmpfilename' was not written!")
		end
%D
%D Generating a PDF directly previously created always a whole page, so we generate EPS first.
%D In recent versions of LilyPond that gets converted to PDF immediately, but we check for that.
%D
%D It would be nice, if \type{-ddelete-intermediate-files} would delete all useless intermediate files
%D that were made for \LATEX-based \type{lilypond-book}, but \type{-dno-aux-files} is needed for that.
%D It would be even better, if mtxrun wouldn't declare non-existing files as \quotation{unchanged}. (Is that fixed now?)
%D
%D Remember: More possible options in \type{lily.scm}.
%D
		os.execute('mtxrun --ifchanged="\lily!tmpfilename" --exec bin:lilypond -dbackend=eps -dno-gs-load-fonts -dinclude-eps-fonts -ddelete-intermediate-files -dno-aux-files -o"\tmpdir/\lily!filename" "\lily!tmpfilename"')
		if not lfs.isfile('\tmpdir/\lily!filename.pdf') then
			os.execute('mtxrun --ifchanged="\tmpdir/\lily!filename.eps" --exec bin:pstopdf "\tmpdir/\lily!filename.eps"')
		end
%D At least on Windows, Lua's \type{os.rename} won't overwrite existing files, so we need to remove them before.
		if lfs.isfile('\lily!img') and lfs.isfile('\tmpdir/\lily!filename.pdf') then
			os.remove('\lily!img')
		end
		os.rename('\tmpdir/\lily!filename.pdf', '\lily!img')
	\stopluacode
%D
%D TODO: Get the relevant dimension directly from LilyPond,
%D to place the instrument name into the left margin for
%D short snippets as well. (Is that possible at all?)
%D
%D If we are not in the middle of some text, we have to check
%D whether LilyPond created an image that is wider than requested:
%D It places the instrument names in the left margin.
%D
	 \ifvmode
	   \getfiguredimensions[\lily!img]%
	   \leavevmode
	   \!!dimenb\figurewidth
	   \ifdim\!!dimenb>\localhsize
	     \!!dimena=\localhsize
	     \advance\!!dimena-\!!dimenb
	     \noindent\hskip\!!dimena
	   \fi
	 \fi
	 \externalfigure[\lily!img]%
	 \egroup}

%D For short snippets, we define an inline alternative to
%D our start/stop pair:

\def\lilypond        {\dosingleempty\dolilypond}
\def\dolilypond[#1]#2{\startlilypond[#1]#2\stoplilypond}

\stopmodule

\protect \doifnotmode{demo}{\endinput}


%D End of file.
%D
%D Here follows a sample. (\type{context t-lilypond --mode=demo})

\starttext

\startbuffer[sample]
\subject{Telemann, TWV 40:11}
\startlilypond
	\relative c' {
		\set Staff.instrumentName = flute
		\key fis \minor
		\time 3/4
		\partial 4
		r8 fis'8 | fis4. cis8 a cis | fis, a cis fis a fis |
		b, d fis b d b | eis,, gis cis eis gis b, |
		a fis' gis, fis' cis eis | fis,4 r8
		a' gis fis | e gis, a e' fis cis |
		d fis, gis b e d | cis4 \trill r8
		cis b a | b dis e gis cis, b |
		a e' dis fis b, a | gis4 \trill r8
	}
\stoplilypond
\stopbuffer

The input
\typebuffer[sample]
is typeset as
\getbuffer[sample]

\type{\startlilypond} accepts options, as in\crlf
\type{\startlilypond[staffsize=18, linewidth=14cm, indent=5cm, time=no, clef=no]}:
\nobreak
\startlilypond[staffsize=18,linewidth=14cm,indent=5cm,time=no,clef=no]
	\relative c' {
		\key fis \minor
		\time 3/4
		\partial 4
		r8 fis'8 | fis4. cis8 a cis | fis, a cis fis a fis |
		b, d fis b d b | eis,, gis cis eis gis b, |
		a fis' gis, fis' cis eis | fis,4 r8
		a' gis fis | e gis, a e' fis cis |
		d fis, gis b e d | cis4 \trill r8
		cis b a | b dis e gis cis, b |
		a e' dis fis b, a | gis4 \trill r8
	}
\stoplilypond

It is also possible to mix text and music:
\lower 8.2pt\hbox{\lilypond[fragment=true,staffsize=16,time=no,clef=no]{g'}} is a g.

TODO: The \type{\lower} in the example above should not be necessary.

\subject{Starlight (round)}

Here is a sample with lyrics:

\startlilypond
<<
\context Staff = onlyone <<
	\clef treble
	\key a \major
	\time 6/8
	\context Voice = one {
		\relative c'' {
			a4.^\markup{1.} e' |
			e8( d cis) b4. |
			e4.^\markup{2.} d4 d8 |
			cis( b) a b4 e,8 |
			a4 a8 gis( a) b |
			cis4 cis8 b( cis d) |
			cis( d) e e4 e,8 |
			fis4 fis8 gis4.
			\bar ":|"
		}
	}
>>
\lyricsto one \new Lyrics {
	\lyricmode {
		Star -- light, star -- bright,
		first star I see to -- night;
		I wish I may, I wish I might
		have the wish I wish to -- night.
	}
}
>>

\stoplilypond

\stoptext
