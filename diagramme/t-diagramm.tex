%D \module
%D   [       file=t-diagramm,
%D        version=2009.01.23, % 2008.09.22
%D          title=\CONTEXT\ User Module,
%D       subtitle=Diagramme,
%D         author=Wolfgang Schuster,
%D           date=\currentdate,
%D      copyright=Wolfgang Schuster,
%D        license=Public Domain]

%M \enableregime[utf]

\writestatus{loading}{Context User Module / Diagramme}

\unprotect

\startinterface all
  \setinterfaceconstant {innercircle} {innercircle}
  \setinterfaceconstant {outercircle} {outercircle}
  \setinterfaceconstant {grid}        {grid}
  \setinterfaceconstant {log}         {log}
\stopinterface

\startinterface all
  \setinterfacevariable {diagramm}    {diagramm}
  \setinterfacevariable {barchart}    {barchart}
  \setinterfacevariable {ringchart}   {ringchart}
  \setinterfacevariable {piechart}    {piechart}
  \setinterfacevariable {spiderchart} {spiderchart}
  \setinterfacevariable {linechart}   {linechart}
  \setinterfacevariable {markerdata}  {markerdata}
  \setinterfacevariable {true}        {true}
  \setinterfacevariable {false}       {false}
\stopinterface

\def\????dm{@@@@dm} % Diagramme
\def\????dg{@@@@dg} % Diagramm
\def\????pt{@@@@pt} % Point

\newcount\diagramm!series
\newcount\diagramm!points

\newdimen\diagrammwidth
\newdimen\diagrammheight
\newdimen\innercircle
\newdimen\outercircle

\def\diagrammvalue#1%
  {\csname\????dg\currentdiagramm#1\endcsname}

%D \macros
%D   {diacatcodes}
%D
%D Die \type {\diacatcodes} Tabelle dient dazu, das Kommentarzeichen innerhalb
%D der Befehle zur Konfiguration der Achsen auf \mono {letter} zu setzen, damit
%D die Formatierung der Zahlen innerhalb des \LUA\ Teils geändert werden kann
%D ohne des Kommentarzeichen erst zu deaktivieren (d.\,h. mittels \type {\%}
%D oder \type {\letterpercent}).

\newcatcodetable \diacatcodes

\startcatcodetable \diacatcodes
    \catcode`\^^M =  5
    \catcode`\    = 10
    \catcode`\\   =  0
    \catcode`\{   =  1
    \catcode`\}   =  2
    \catcode`\%   = 11
    \catcode`\!   = 11
    \catcode`\?   = 11
\stopcatcodetable

%D \macros
%D   {startdiagramm,stopdiagramm}
%D
%D Diagramme werden innerhalb der Diagramm||Umgebung gesetzt, der erste
%D Parameter ist der Name des Diagrammtyp und zwingend notwendig, der
%D zweite Parameter kann benutzt werden um lokale Einstellungen vorzunehmen.
%D
%D In Dokumenten bietet es sich an anstatt der Diagramm||Umgebung die
%D mit \type {\definediagramm} erstellte Umgebung zu verwenden.

\def\dostartdiagramm[#1][#2]%
  {\bgroup
   \ifsecondargument
     \setupdiagramm[#1][#2]%
   \fi
   \edef\currentdiagramm{\csname\????dg#1\c!type\endcsname}%
   \diagrammwidth =\diagrammvalue\c!width      \relax
   \diagrammheight=\diagrammvalue\c!height     \relax
   \outercircle   =\diagrammvalue\c!outercircle\relax
   \innercircle   =\diagrammvalue\c!innercircle\relax
   \ctxlua{thirddata.diagramm.initialization()}%
   \ctxlua{thirddata.diagramm.setup = { alternative = "\diagrammvalue\c!alternative", grid = "\diagrammvalue\c!grid" }}}

\def\dostopdiagramm
  {\startMPcode
   \ctxlua{thirddata.diagramm.drawchart("\currentdiagramm")}%
   \stopMPcode
   \egroup}

%D \macros
%D   {startdata,stopdata}
%D
%D Datenblock, setzte aller Zähler zurück.

\def\startdata
  {\global\diagramm!series\zerocount
   \global\diagramm!points\zerocount
   \ctxlua{thirddata.diagramm.data = { }}%
   \def\point{\dosingleargument\diagrammpoint}}

\let\stopdata\relax

%D \macros
%D   {startseries,stopseries}
%D
%D Datenserie, erhöhe den Zähler für die Serie um eins und setze den Zähler
%D für den Datenwert zurück.

\def\startseries
  {\dosingleempty\dostartseries}

\def\dostartseries[#1]%
  {\bgroup
   \getparameters[\????pt][\c!color=\diagrammvalue\c!color,#1]%
   \global\diagramm!points\zerocount
   \global\advance\diagramm!series\plusone
   \ctxlua{thirddata.diagramm.series(\number\diagramm!series)}}

\def\stopseries
  {\egroup}

%D \macros
%D   {point,diagrammpoint}
%D
%D Datenwert, speichert die Werte in einer Tabelle zur späteren Nutzung ab.
%D Um Konflikte mit anderen Makros zu vermeiden wird die Kuzform \type {\point}
%D nur innerhalb des Diagrammes definiert.

\def\diagrammpoint[#1]%
  {\bgroup
   \global\advance\diagramm!points\plusone
   \getparameters[\????pt][\c!y=1,#1]%
   \ctxlua{thirddata.diagramm.point(\number\diagramm!series,\number\diagramm!points,\@@@@pty,"\@@@@ptcolor")}%
   \egroup}

%D \macros
%D   {definediagramm}
%D
%D Neue Diagrammtypen werden mit den \type {\definediagramm} Befehl erstellt,
%D dabei werden einzelne Werte bereits mit Vorgaben besetzt, die vom Benutzer
%D geändert werden können.
%D
%D Zusätzlich wird eine eigene Umgebung für den Diagrammtyp erstellt,
%D welcher einen optionalen Parameter für Einstellungen bietet.

\def\definediagramm
  {\dodoubleempty\dodefinediagramm}

\def\dodefinediagramm[#1][#2]%
  {\setupdiagramm
     [#1]
     [\c!type=\v!barchart, % default for \startdiagramm
      \c!width=\@@@@dmwidth,
      \c!height=\@@@@dmheight,
      \c!outercircle=\@@@@dmoutercircle,
      \c!innercircle=\@@@@dminnercircle,
      \c!color=\@@@@dmcolor,
      \c!background=\@@@@dmbackground,
      \c!backgroundscreen=\@@@@dmbackgroundscreen,
      \c!backgroundcolor=\@@@@dmbackgroundcolor,
      \c!alternative=\@@@@dmalternative,
      \c!format=\@@@@dmformat,
      \c!grid=\@@@@dmgrid,
      \c!frame=\@@@@dmframe,
      #2]%
   %\setvalue{\e!setup#1\e!endsetup}{\setupdiagramm[#1]}%
   \setvalue{\e!start#1}{\dodoubleempty\dostartdiagramm[#1]}%
   \setvalue{\e!stop #1}{\dostopdiagramm}}

%D \macros
%D   {setupdiagramm,setupdiagramme}
%D
%D Die Einstellungen für die einzelnen Diagrammtypen können mittels
%D des \type {\setupdiagramm} Befehl konfiguriert werden, um Werte für
%D alle verschiedenen Typen gleichzeitig zu setzen kann der
%D \type {\setupdiagramme} Befehl verwendet werden.

\def\setupdiagramm
  {\dodoubleargument\dosetupdiagramm}

\def\dosetupdiagramm[#1][#2]%
  {\getparameters[\????dg#1][#2]}

\def\setupdiagramme
  {\dodoubleargument\getparameters[\????dm]}

%D \macros
%D   {xrange,yrange}
%D
%D Range

\def\????ra{@@@@ra}

\def\xrange[#1]%
  {\getparameters[\????ra\c!x][#1]%
   \ctxlua{thirddata.diagramm.xrange("\@@@@raxmin","\@@@@raxmax")}}

\def\yrange[#1]%
  {\getparameters[\????ra\c!y][#1]%
   \ctxlua{thirddata.diagramm.yrange("\@@@@raymin","\@@@@raymax")}}

%D \macros
%D   {xaxis,yaxis,setupaxis}
%D
%D Achsen

\def\????ax{@@@@ax}

\def\xaxis
  {\pushcatcodetable
   \setcatcodetable\diacatcodes
   \doxaxis}

\def\doxaxis[#1]%
  {\getparameters[\????ax\c!x][#1]%
   \popcatcodetable
   \ctxlua{thirddata.diagramm.xaxis("\@@@@axxmin","\@@@@axxmax")}%
   \ctxlua{thirddata.diagramm.setup.xaxis   = true              }%
   \ctxlua{thirddata.diagramm.setup.xtick   = true              }%
   \ctxlua{thirddata.diagramm.axis.x.log    = \@@@@axxlog       }%
   \ctxlua{thirddata.diagramm.axis.x.format = "\@@@@axxformat"  }}

\def\yaxis
  {\pushcatcodetable
   \setcatcodetable\diacatcodes
   \doyaxis}

\def\doyaxis[#1]%
  {\getparameters[\????ax\c!y][#1]%
   \popcatcodetable
   \ctxlua{thirddata.diagramm.yaxis("\@@@@axymin","\@@@@axymax")}%
   \ctxlua{thirddata.diagramm.setup.yaxis   = true              }%
   \ctxlua{thirddata.diagramm.setup.ytick   = true              }%
   \ctxlua{thirddata.diagramm.axis.y.log    = \@@@@axylog       }%
   \ctxlua{thirddata.diagramm.axis.y.format = "\@@@@axyformat"  }}

\def\setupaxis
  {\pushcatcodetable
   \setcatcodetable\diacatcodes
   \dodoubleargument\dosetupaxis}

\def\dosetupaxis[#1][#2]%
  {\def\docommand##1{\getparameters[\????ax##1][#2]}%
   \popcatcodetable
   \processcommalist[#1]\docommand}

\setupaxis
  [\c!x,\c!y]
  [\c!log=\v!false,
   \c!format=%d]

%D Für alle Diagramme werden einige Einstellungen vorgenommen,
%D die vom Benutzer innerhalb des Dokumentes geändert werden können.

\setupdiagramme
  [\c!width=.6\textwidth,
   \c!height=.4\textwidth,
   \c!outercircle=\dimexpr\diagrammvalue\c!height\relax,
   \c!innercircle=\dimexpr\diagrammvalue\c!outercircle*3/4\relax,
   \c!color=\s!black,
   \c!background=,
   \c!backgroundscreen=,
   \c!backgroundcolor=,
   \c!alternative=\v!a,
   \c!format=\v!tex,
   \c!grid=\v!no,
   \c!frame=\v!off]

%D Erstellen der einzelnen Diagrammtypen

\definediagramm[\v!diagramm   ]
\definediagramm[\v!barchart   ][\c!type=\v!barchart   ]
\definediagramm[\v!piechart   ][\c!type=\v!piechart   ]
\definediagramm[\v!ringchart  ][\c!type=\v!ringchart  ]
\definediagramm[\v!spiderchart][\c!type=\v!spiderchart]
\definediagramm[\v!linechart  ][\c!type=\v!linechart  ]
\definediagramm[\v!markerdata ][\c!type=\v!markerdata ]

\ctxloadluafile{t-diagramm}{}

\protect \endinput
